\section{Problem B}
\subsection{Statement}
\href{http://foobar.iiitd.edu.in/contest/team/problem.php?id=528}{click}

\subsection{Solution}
Let's consider the leftmost detector, say $P_x, C_x$. Now, all calls that it detects
must start to its left, and end to its right. Since we are trying to minimise the total number
of calls, it makes no sense to have any calls starting and ending to its left, since they
will not be detected by any detector.
So, we want to start exactly $C_x$ calls to its left, and have them end to its right.

But what about where to end these calls? Clearly, since we want to minimise the total number
of calls, it makes the most sense to end them as late as possible (so they can trigger
as many detectors as possible). This is, of course, subject to the number of calls detected
by each detector, otherwise we would just end them all on the right most house.

What about the next left-most detector, say $P_y, C_y$?
\begin{itemize}
\item If we have $C_y < C_x$, then clearly some of the $C_x$ calls started before $P_x$
must end before $P_y$. How many? Exactly $C_x - C_y$, since the rest can safely end later
without violating this detector's data. Note that in this case, no extra calls are started.

\item If we have $C_y > C_x$, then we need to start some additional calls so we can meet this
detector's data. How many? Exactly $C_y - C_x$. Note that in this case, this number \emph{does}
add to the total number of calls, and this is the only scenario where the total increases.
\end{itemize}

NB: we have an algorithm that requires us to iterate from the left most detector, to the 
rightmost detector. We can do this by first sorting the detectors by $P$. Any sort which is
$O(N\log N)$ should suffice. In C++, this is quite easy to do, by making an array
(or \verb|std::vector|) of \verb|std::pair| and then calling \verb|std::sort| with the right
bounds, since the default comparison for \verb|std::pair| is to compare the first element, and
break ties using the second. 